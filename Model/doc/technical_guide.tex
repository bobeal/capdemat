\documentclass[11pt,a4paper]{report}
\usepackage[latin1]{inputenc}
\usepackage{url}

\author{Benoit Orihuela}
\title{CVQ Technical Documentation}
\pagestyle{headings}
\frenchspacing

\begin{document}
\maketitle
\tableofcontents
\newpage

\chapter{Glossary}
\begin{itemize}
  \item DO : Domain Object
  \item DAO : Data Access Object
\end{itemize}

\chapter{Introduction}
This document is aimed at providing a technical view of the CVQ Model part.
It mainly deals with :
\begin{itemize}
  \item the architecture (Spring)
  \item the persistence layer
  \item the access control policy
  \item the configuration of the application
  \item the external services
\end{itemize}

\chapter{Extending the Model}

\section{Adding a new request type}
\begin{itemize}
  \item create at least the corresponding class in the general request mapping file
  (\emph{src/xml/hibernate/request.hbm.xml} 
  \item if you added a new Hibernate mapping file, add it in the required files :
    \begin{itemize}
      \item \emph{script/generate\_beans.sh}
      \item \emph{build.xml}
      \item \emph{conf/spring/applicationContext.xml}
    \end{itemize}
  \item run the script \emph{script/generate\_beans.sh} to create and/or update
  the DOs
  \item if needed, create the DAOs in \emph{src/java/fr/cg95/cvq/service/dao}
  and \emph{src/java/fr/cg95/cvq/service/dao/hibernate}
  \item if you have special needs for SQL requests associated with this request,
  create a new DAO to take it into account
  \item create a service interface and implementation in respectively
  \emph{src/java/fr/cg95/cvq/service} and
  \emph{src/java/fr/cg95/cvq/service/spring} (override at least the
  \emph{create} method to set your associations properly)
  \item add an unique string identifier for the new request in
  \emph{src/java/fr/cg95/cvq/service/IRequestService.java} 
  \item add en entry for the new service in the Spring application context file
  (\emph{conf/spring/applicationContext.xml}) and add the necessary DAOs and
  services for the service
  \item code the business logic in the Spring service
  \item write an unit test in \emph{test/java/fr/cg95/cvq/service} and your
  service in transaction part of the \emph{applicationContext-test.xml} file
  \item add your new service in
  \emph{test/java/fr/cg95/cvq/testtool/ServiceTestCase.java} 
  \item add an entry for the new request type in each city's
  \emph{request_type.properties} file
\end{itemize}

\section{Adding a new city}
To add a new city in the Front Office or Back Office webapp :
\begin{itemize}
  \item add a session factory entry in the
  \emph{applicationContext-deployment.xml} file
  \item add an entry pointing to the new session factory in the security context
  service in the \emph{applicationContext.xml} file
\end{itemize}

\chapter{External Services}

\section{Payline}

\section{Wynid}

\section{Horanet}

\chapter{Frequently Asked Questions}

Q: I have modified an Hibernate mapping file, how can I re-generate the
   business objects ?

A: Use the script generate_beans.sh located in the script
   directory. If you have added an new mapping file, you'll have to add
   it manually in the script file

Q: I have added a new Hibernate mapping file. What do I have to modify
   for it to be taken into account ?

A: Add an entry corresponding to the new mapping file into the
   following files : build.xml, applicationContext-hibernate.xml (into
   conf/spring directory) and generate_beans.sh (into script directory)

Q: I have added a new service method and a new session is created for
   every DAO request I do in it (deprecated by OpenSessionInView pattern).

A: This method has a name not recognized by the transaction
   manager. Add something for it in applicationContext-hibernate.xml

Q: I want to setup the DB manually. What I have to do ?

A: First, create the database : create database "cartevaloise" with
   encoding='latin1'. Then, change to the db directory, create the tables
   (psql cartevaloise < schema-export.sql), then initialize the
   referential data with the scripts provided in script/ directory

\end{document}

